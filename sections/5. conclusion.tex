\section{Discussion}
We discuss our results in relation to the three sub-research questions, for each type of stakeholder: which type of explanation is most suited, what makes these explanations most suitable, and can different explanation types be meaningfully combined. We discuss each research question in turn. 

\subsection{SQ1: What type of explanation is most suited for the different stakeholders?}

When analyzing the results regarding preferences for the different stakeholders, we notice that \textit{candidates} prefer short, clearly structured, straightforward texts, which allow them to quickly browse and judge the vacancies. These texts should include what they consider the most important information: features like travel distance, salary, and minimum requirements. This information should be central and easy to identify, preferably in bullets. \textit{Recruiters} also prefer texts, but disagree on the amount of text that is required. Thus, a short text, that centrally mentions potential `deal-breakers' and gives the main few arguments to motivate why a match was made should be the default. However, recruiters also prefer to have access to a more exhaustive text, which can provide them with more material that could be used to convince both parties (candidate and company) to agree with the match. On the contrary, \textit{company representatives} prefer graph-based explanations, as those assist them in quickly getting an overview of even more complex explanations at a glance. Such a representation also allows them to quickly scan different information, while reading would require more time and effort. These results are somewhat in line with our hypotheses H1a and H1b, but not entirely. While one of the candidates did have a lot of knowledge of AI, she still preferred the textual explanation. On the other hand, both company representatives did not have a strong background in AI, but preferred the graph-based explanations. We argue that it is not the AI knowledge per se that makes the graph preferable, but the amount of experience with, and affinity towards, reading graphs and charts.

\subsection{SQ2: What aspects of explanations make them more suited for each stakeholder?}

We also looked more specifically at the motivation for why certain stakeholders prefer certain explanations. For \textit{candidates}, the textual explanations were largely preferred due to their simplicity, and because they felt more `personal'. In particular, they preferred texts using simple language, that is clearly structured and not longer than a few short paragraphs. \textit{Recruiters} also preferred the textual explanations, due to their simplicity compared to the graph- and feature-based explanations. In particular, they struggled to interpret the visualizations and felt quite overwhelmed due to their `math-heavy' nature. Furthermore, the text directly assists them in their day-to-day tasks, as they can almost use some of the paragraphs verbatim to try and convince companies and candidates of the adequacy of a match. Finally, for \textit{company representatives}, the graph-based explanations were preferred, largely due to their ability to make more complex, high-level connections, within the data visible at a glance. Within the textual explanations, it became difficult for them to figure out the full line of reasoning of the model, due to there being a lot of `steps' from the vacancy to the candidate, which made it hard to process. The bar chart also made the text more accessible, but the graph-based explanation was considered a better option.

\subsection{SQ3: In what way can different explanation types be combined to make them useful for each stakeholder?}

Finally, we evaluated the stakeholders' preferences in terms of hybrid explanations, indicating how to combine different explanations together. Our results highlight how the feature-based explanation was poorly received by all stakeholders; however, they also indicated it to have potential, in case it is used to support one of the other explanation types. For both \textit{candidates and recruiters}, the textual explanations should be the center of the explanation, by default in its simplified form. The user should then have the possibility to access additional information, using a toggle to get a more descriptive version of the text. Considering the difficulty of conveying relative importance levels within a text, the feature attribution map can be linked (i.e., through matching color coding) to clarify the text. The graph-based explanation can then be used as an optional addition, in case the text itself is not clear enough, or if the user wants even more evidence for a specific suggestion. 

On the contrary, \textit{company representatives} prefer the simplified graph to be central, supported by a textual explanation. Within the graph, most details (e.g., exact values) should be made optional, so that it is not overwhelming: they mainly want to be able to quickly parse the most critical paths in the graph, and only look at details when necessary. Additionally, a bar chart indicating how important different feature types were, could be used to complement the text, in order to help them focus their attention on the paragraphs touching on those feature types. 

% \begin{itemize}
%     \item \textbf{SQ1:}
%     \begin{itemize}
%         \item Candidates prefer short, clearly structured, straightforward texts that allow them to quickly browse vacancies. The most important information, such as travel distance, minimum requirements, etc. should be central, and preferably in bullets.  
%         \item Recruiters also prefer texts, but disagree on the amount of text that is required. Thus, a short text, that centrally mentions potential `deal-breakers' and gives the main few arguments for why a match was made should be the default. It should also be possible for them to access a more exhaustive text, as that provides them with more subject matter to convince both parties to agree. 
%         \item Company representatives prefer graph-based explanations, as those assist them in quickly getting an overview of even more complex explanations at a glance. Reading also requires more time and effort than quickly scanning a graph. 
%         \item
%         \item this is somewhat in line with our hypotheses (1a, 2b), but not entirely. While one of the candidates did have a lot of knowledge of AI, she still preferred the textual explanation. On the other hand, the company representatives both did not have a background in AI, but preferred the graph. Ergo, it is not the AI knowledge per se that makes the graph preferable, but the amount of experience with- and affinity towards, reading graphs.
%     \end{itemize}
%     \item \textbf{SQ2:}
%     \begin{itemize}
%         \item For candidates, the textual explanations were largely preferred due to their simplicity, and them feeling more `personal'. Especially texts that use easy language, are clearly structured, and are not too long, were indicated to be highly useful to them. 
%         \item For recruiters, the textual explanations were also preferred due to their simplicity compared to the graph- and bar chart explanations. They struggled to read the visualizations and felt quite overwhelmed due to their `math-heavy' nature. Furthermore, the text very directly assists them in their day-to-day tasks, as they can almost use some of the paragraphs verbatim to try and convince companies/candidates of the adequacy of a match.
%         \item For company representatives, the graph was preferred, largely due to its ability to make more complex, high-level connections within the data visible at a glance. Within the textual explanations, it became difficult to them to figure out the full line of reasoning of the model, due to there being a lot of `steps' from the vacancy to the candidate. This made it hard to process. The bar chart did make the text more accessible, but the graph was still superior. 
%     \end{itemize}
%     \item \textbf{SQ3:}
%     \begin{itemize}
%         \item Considering the bar chart explanation was poorly received by all stakeholders, but indicated to have potential, it should merely be used as support for one of the other explanation types. 
%         \item For candidates and recruiters, the textual explanations should be the center of the explanation, by default in its simplified form. In case the user then wants more information on the explanation, they can use a toggle to get a more descriptive version of the text. Considering the difficulty of conveying relative importance levels within a text, the feature attribution map can be linked (i.e., matching color-coding) to clarify the text. The graph can then be used as an optional addition, in case the text itself is not clear enough, or the user wants even more evidence. 
%         \item Company representatives want the simplified graph to be central, supported by a textual explanation. Within the graph, most details (e.g., exact values) should be made optional, so that it is not overwhelming. I.e., they mainly want to be able to quickly parse the most important paths in the graph, and only look at details when necessary. Additionally, having a bar chart of how important different feature types were that supports the text, can help them focus their attention on the paragraphs touching on those feature types. 
%     \end{itemize}
% \end{itemize}



% \begin{itemize}
%     \item Different stakeholders indeed have different preferences
%     \item Still, the best representation also depends on the individuals, so leaving the possibility of any of the options for all stakeholders is a good idea
%     \item for different stakeholders, different aspects should be central, and some information is more essential. 
%     \item in general: candidates and recruiters want the text to be central, for candidates it should be more brief, while for recruiters it can go into more detail. Having a toggle between brief and detailed is desirable, as preferences differ, even within stakeholder types. The text should allow easy scrutiny (putting deal-breakers centrally), and easy verification (allowing them to access CVs and vacancy texts easily). This text should be supported by the graph, in case there is still doubt left after reading the text. However, the simplified version of the graph should always be used as a starting point, where unimportant paths have been removed. 
%     \item Company reps want the simplified graph to be central, supported by some text. Additionally, having a bar chart of how important different feature types were that supports the text, can help them focus their attention on the paragraphs touching on those feature types. 
% \end{itemize}




\subsection{Limitations and future work}
A key limitation to acknowledge is the relatively small sample size we used. Considering we only interviewed two individuals from each stakeholder type, it is possible that some of our results are based on their personal biases, which may not be representative of the entire population. We attempted to minimize these biases through careful selection of participants, making sure to include individuals from different backgrounds, both in terms of their expertise and personal characteristics. This focused sample did allow us to focus on the quality of the data we collected; due to the limited number of participants, it was feasible to interview them for longer periods of time. Future research should aim to make use of the validated interview guide with a larger sample. By making use of the guidelines we provided on how to represent automatically-generated explanations, it should be possible to design a single, hybrid explanation, that can be evaluated more quickly. As a result, interviews will take less time, making it more practical to use a larger sample size. 

Furthermore, we acknowledge room to improve the model we used to generate explanations, and that it may therefore not have generated the most sensible explanations. To counteract this behavior, we manually selected explanations that seemed suitable for the interviews. %However, the possibly lower quality of these explanations may have had an impact on how the participants perceived the different explanation types. 
Future work could therefore use the provided interview guide to evaluate and compare explanations generated using a number of different techniques (e.g., attention mechanisms, saliency, post hoc methods). Different model architectures could also be compared in order to determine which architectures generate better explanations (either for a specific evaluation objective, or as a whole). 


\subsection{Conclusion}
In this paper, we aimed to develop and validate an interview guide for determining the explanation preferences of different stakeholder types. Additionally, we aimed to establish guidelines for creating XAI-generated explanations for different stakeholders within the field of job recommendation. The interview guide was largely proven to be adequate for determining the preferences of different stakeholders; a few minor changes were made to it in order to attain more concrete responses from the participants. Through the use of the interview guide, we found that candidates prefer explanations to take the shape of a short, clearly-structured text, that centrally contains the most crucial information. Recruiters, on the other hand, also preferred textual explanations, but were less strict on it having to be brief - indicating that longer texts could be useful in some scenarios. Company representatives indicated a preference towards graph-based explanations, as those allowed them to get a comprehensive overview of even more complex explanations. 